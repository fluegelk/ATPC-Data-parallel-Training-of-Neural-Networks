\documentclass[conference,compsoc,a4paper]{IEEEtran}
\usepackage[utf8]{inputenc}


% configs from ba
\usepackage[english]{babel}
\usepackage[raiselinks=true,
			bookmarks=true,
			bookmarksopenlevel=1,
			bookmarksopen=true,
			bookmarksnumbered=true,
			hyperindex=true,
			plainpages=false,
			pdfpagelabels=true,
			pdfborder={0 0 0.5},
			colorlinks=false,
			linkbordercolor={0 0.61 0.50},
			citebordercolor={0 0.61 0.50}]{hyperref}
\addto\extrasenglish{%
  \def\chapterautorefname{Chapter}%
  \def\chapterrefname{Chapter}%
  \def\sectionautorefname{Section}%
  \def\sectionrefname{Section}%
  \def\subsectionautorefname{Section}%
  \def\subsectionrefname{Section}%
  \def\subsubsectionautorefname{Section}%
  \def\subsubsectionrefname{Section}%
  \def\paragraphautorefname{Paragraph}%
  \def\paragraphrefname{Paragraph}%
}

\ifCLASSOPTIONcompsoc
  % IEEE Computer Society needs nocompress option
  % requires cite.sty v4.0 or later (November 2003)
  \usepackage[nocompress]{cite}
\else
  % normal IEEE
  \usepackage{cite}
\fi


% correct bad hyphenation here
\hyphenation{op-tical net-works semi-conduc-tor}


\begin{document}

\title{Data-parallel Training of Neural Networks}


% author names and affiliations
% use a multiple column layout for up to three different
% affiliations
\author{\IEEEauthorblockN{Katharina Flügel}
\IEEEauthorblockA{TODO?}}


% make the title area
\maketitle

% General rule: no math, special symbols or citations in the abstract
\begin{abstract}
TODO Abstract
\end{abstract}

\section{Introduction} % (fold)
\label{sec:introduction}
% Motivation
% Hypthesis and overview over results
% Organisation of this paper

% section introduction (end)

\section{Preliminaries} % (fold)
\label{sec:preliminaries}
%TODO: look at preliminaries in \cite{paine2013-GPU-async-SGD} and \cite{chilimbi2014-Project-Adam}

% Deep Learning definition:
% \begin{itemize}
% 	\item use many layers
% 	\item learning hierarchical features from the raw input data instead of using hand-crafted features \cite{chilimbi2014-Project-Adam}
% \end{itemize}

This sections gives a short introduction to machine learning using convolutional neural networks, defining all necessary concepts used in the remainder of this paper.
More in-depth information can, for example, be found in \cite{goodfellow2016-DeepLearningBook}.

Neural networks consist of multiple neurons.
A neuron has some weighted connections, for example to previous neurons, with weights $w$ and a bias $b$.
It receives some input $x$ via its connections and computes a weighted sum $w^Tx+b$ which is then passed to an \emph{activation function} $\phi(\cdot)$.
Typical activation functions are $\tanh$, the logistic function, or more recently the rectified linear unit (ReLU).
The output $\phi(w^Tx+b)$ of the neuron can then be passed as input to the next neuron.

Multiple neurons can be organized into \emph{layers} where the output of each layer is passed as input to the next.
The input data -- for example the pixel values of a training image -- is passed to the first layer.
The network output is encoded in the last layer's output.
%
There are different types of layers depending on how the neurons are connected to the previous layer.
In computer vision and image classification, the most important types of layers are fully-connected layers, convolutional layers, and pooling-layers.

In \emph{fully-connected layers}, each neuron of the first layer is connected to each neuron in the second layer.
These layers consist of many connections which result in numerous parameters.
Fully-connected layers are typically used to determine the image class once the features have been extracted.

\emph{Convolutional layers} are based on the convolutional filtering used in digital image processing, for example in the Gaussian blur filter TODO cite? or edge detection filters like Sobel or Prewitt TODO cite?.
They use prior knowledge, for example the data hierarchy TODO cite, to reduce the number of parameters.
Convolutional layers consist of a small filter -- usually between 3x3 and 7x7 pixels -- that is shifted over the image.
At each pixel, the weighted sum of its surrounding pixels is computed and passed to the activation function.
Instead of connecting individual neurons with weighted connections, the filter weights are therefore reused for each pixel.
This reduces the number of parameters significantly, which in turn speeds up the training.
Convolutional layers are often used together with pooling layers as features extractors TODO cite.

\emph{Pooling layers} aggregate adjacent pixels to a single output pixel using either the sum, the average or the maximum of the pooled pixels.
Similarly to convolutional layers, the pooling operation is shifted over the image and apply repeatedly the the different parts of the image.
Pooling layers reduces the spatial size of the input image and generally improve the models robustness to noise and clutter in the input.
TODO look at Y-L. Boureau, J. Ponce, Y. LeCun. A Theoretical Analysis of Feature Pooling in Visual Recognition. ICML 2010

A \emph{loss function} is used to evaluate the accuracy by comparing the networks output to the actual, expected output.
Different loss functions for different types of problems exist.
For image classification, a typical loss function is the cross entropy loss.
%
In general, neural networks can be thought of as ``large parameterized function[s]'' \cite{paine2013-GPU-async-SGD} where all the weights and biases are model parameters that can be modified to improve the accuracy.
When training a neural network, we aim to minimize the loss by adjusting its parameters.

A widely used optimization strategy for training neural networks is \emph{stochastic gradient descent (SGD)}. % TODO cite?
Gradient descent is an optimization algorithm to find local minima by adjusting the parameters in the opposite direction of the gradient.
%
To compute the gradient of a neural network, we typically use the \emph{backpropagation} algorithm.
Backpropagation is based on the chain rule, a rule used to compute the derivative of function compositions.
%
Instead of computing the gradient over the entire training set at once, SGD partitions the shuffled training set into \emph{batches}.
For each batch, we compute an averaged gradient using back-propagation and update the model parameters accordingly.
%
A \emph{training epoch} consists of processing each batch in the training set once.
Usually, multiple epochs -- that is multiple passes through the training set -- are required to train a neural network to convergence \cite{chilimbi2014-Project-Adam}.

There are many different hyper-parameters to tune the neural network and its training.
In this paper, only the \emph{batch size}, that is the number of training instances per batch, is of importance.
A larger batch size generally reduces the training time per epoch while increasing the number of epochs required to reach a certain accuracy.
As we will see in \autoref{sec:parallelizing_neural_networks} and \ref{sec:experimental_results}, the batch size can also affect the scalability of some parallelization techniques.

% TODO position?
A dataset consists of two separate sets, one for training the network and one to test its accuracy an unseen inputs and to avoid overfitting.
Each of these sets contains pairs of an input instance and the correct label.
In this paper, we focus on the image classification problem, where given an image, the neural network shall correctly determine the class depicted in the image.

% TODO additional? GPUs for neural networks and why they work well, short MPI introduction

% section preliminaries (end)

\section{Related Work} % (fold)
\label{sec:related_work}

% section related_work (end)

\section{Parallelizing Neural Networks} % (fold)
\label{sec:parallelizing_neural_networks}

\subsection{Model Parallelism} % (fold)
\label{sub:model_parallelism}
% definition/how does it work
% pro/contra (compared to data-parallelism)

% subsection model_parallelism (end)

\subsection{Data Parallelism} % (fold)
\label{sub:data_parallelism}
% definition/how does it work in general
% different approaches: synchronous/asynchronous etc.
% pro/contra (compared to model-parallelism)

% subsection data_parallelism (end)

\subsection{Hybrid Approaches} % (fold)
\label{sub:hybrid_approaches}
% definition/how does it work
% motivation, pro/contra to just one parallelism technique

% subsection hybrid_approaches (end)

% section parallelizing_neural_networks (end)


% for implementation: why did I choose synchronous data parallelism with all-reduce? what's the advantage for parallel programming

\section{Implementation} % (fold)
\label{sec:implementation}
% used tools and libraries (incl. version number)
% execution environment/hardware
% interesting/relevant implementation details

% section implementation (end)

\section{Experimental Results} % (fold)
\label{sec:experimental_results}
% illustrate results with plots etc.
% describe and interpret results

% section experimental_results (end)

\section{Conclusion} % (fold)
\label{sec:conclusion}
% summarize contributions, reference to introduction/abstract
% summarize central results and implications
% revisit section key points
% Benefits and shortcomings of approach and results

% section conclusion (end)


% \begin{thebibliography}{1}
% % TODO
% \end{thebibliography}
\bibliographystyle{IEEEtran}
\bibliography{IEEEabrv,references}

\end{document}

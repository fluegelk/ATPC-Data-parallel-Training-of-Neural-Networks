\documentclass[conference,compsoc,a4paper]{IEEEtran}
\usepackage[utf8]{inputenc}


% configs from ba
\usepackage[english]{babel}
\usepackage[raiselinks=true,
			bookmarks=true,
			bookmarksopenlevel=1,
			bookmarksopen=true,
			bookmarksnumbered=true,
			hyperindex=true,
			plainpages=false,
			pdfpagelabels=true,
			pdfborder={0 0 0.5},
			colorlinks=false,
			linkbordercolor={0 0.61 0.50},
			citebordercolor={0 0.61 0.50}]{hyperref}
\addto\extrasenglish{%
  \def\chapterautorefname{Chapter}%
  \def\chapterrefname{Chapter}%
  \def\sectionautorefname{Section}%
  \def\sectionrefname{Section}%
  \def\subsectionautorefname{Section}%
  \def\subsectionrefname{Section}%
  \def\subsubsectionautorefname{Section}%
  \def\subsubsectionrefname{Section}%
  \def\paragraphautorefname{Paragraph}%
  \def\paragraphrefname{Paragraph}%
}

\ifCLASSOPTIONcompsoc
  % IEEE Computer Society needs nocompress option
  % requires cite.sty v4.0 or later (November 2003)
  \usepackage[nocompress]{cite}
\else
  % normal IEEE
  \usepackage{cite}
\fi


% correct bad hyphenation here
\hyphenation{op-tical net-works semi-conduc-tor}


\begin{document}

\title{Data-parallel Training of Neural Networks}


% author names and affiliations
% use a multiple column layout for up to three different
% affiliations
\author{\IEEEauthorblockN{Katharina Flügel}
\IEEEauthorblockA{TODO?}}


% make the title area
\maketitle

% General rule: no math, special symbols or citations in the abstract
\begin{abstract}
TODO Abstract
\end{abstract}

\section{Introduction} % (fold)
\label{sec:introduction}
% Motivation
% Hypthesis and overview over results
% Organisation of this paper

% section introduction (end)

\section{Preliminaries} % (fold)
\label{sec:preliminaries}
% short intro to neural networks:
% - short motivation
% - short description of the training (forward, backward)
% - FC vs. Conv-Layers


% section preliminaries (end)

\section{Related Work} % (fold)
\label{sec:related_work}

% section related_work (end)

\section{Parallelizing Neural Networks} % (fold)
\label{sec:parallelizing_neural_networks}

\subsection{Model Parallelism} % (fold)
\label{sub:model_parallelism}
% definition/how does it work
% pro/contra (compared to data-parallelism)

% subsection model_parallelism (end)

\subsection{Data Parallelism} % (fold)
\label{sub:data_parallelism}
% definition/how does it work in general
% different approaches: synchronous/asynchronous etc.
% pro/contra (compared to model-parallelism)

% subsection data_parallelism (end)

\subsection{Hybrid Approaches} % (fold)
\label{sub:hybrid_approaches}
% definition/how does it work
% motivation, pro/contra to just one parallelism technique

% subsection hybrid_approaches (end)

% section parallelizing_neural_networks (end)


% for implementation: why did I choose synchronous data parallelism with all-reduce? what's the advantage for parallel programming

\section{Implementation} % (fold)
\label{sec:implementation}
% used tools and libraries (incl. version number)
% execution environment/hardware
% interesting/relevant implementation details

% section implementation (end)

\section{Experimental Results} % (fold)
\label{sec:experimental_results}
% illustrate results with plots etc.
% describe and interpret results

% section experimental_results (end)

\section{Conclusion} % (fold)
\label{sec:conclusion}
% summarize contributions, reference to introduction/abstract
% summarize central results and implications
% revisit section key points
% Benefits and shortcomings of approach and results

% section conclusion (end)


% \begin{thebibliography}{1}
% % TODO
% \end{thebibliography}
\bibliographystyle{IEEEtran}
\bibliography{IEEEabrv,references}

\end{document}
